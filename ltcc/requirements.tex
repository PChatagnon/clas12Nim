\section{Requirements}

The LTCC requirements to allow for an adequate pion/kaon discrimination include:

\begin{itemize}
	\item Maximizing the polar angle coverage in each of the six sectors up to angles of 30\mdeg;
	\item Minimizing the radiation length in the active area of CLAS12;
	\item Fitting the LTCC modules in the available space between the Drift Chambers~\cite{dc-nim} and the Forward
          Time-Of-Flight system~\cite{ftof-nim};
	\item Producing a signal for pions in the momentum range $4-8$~GeV.
\end{itemize}

The radiation length of the detector was minimized in the original CLAS design by placing the light collecting cones
and PMTs in the regions obscured by the torus magnet coils. In the active area the window radiation length is 0.02\%.
However, the achieved azimuthal and polar angle coverage for these counters has been slightly reduced in CLAS12
as the distance between the target and the LTCC was increased by about 2~m compared to the CLAS configuration.
This brought some of the passive elements of the LTCC into the active area of the detectors behind it, namely the
support structure of the mirrors, the Winston cones, the PMT magnetic shields, and the detector walls. The remaining
requirements to allow for adequate pion/kaon separation of the LTCC system have been addressed by the refurbishment
and are discussed in this paper.


