\section{Calibration}

Since the SVT modules are designed with a binary readout system, the analog channel response cannot be measured
directly. Instead, the analog response is reconstructed by injecting a calibration charge on the channel and measuring
the corresponding occupancy over a range of threshold values. 

\begin{figure}[hbt] 
	\centering 
	\includegraphics[width=1.0\columnwidth,keepaspectratio]{threshold-scan.jpg}
	\caption{Threshold scan on a single representative SVT channel. Three occupancy plots taken at different
          pulser amplitudes and the response plot (lower right) are shown. The parameters results of the $erfc$ and
          linear fits are shown on the plots.}
	\label{fig:threshold-scan}
\end{figure}

The output signals from the FSSR2 chip can be converted to charge using either internal or external calibration
pulses. Because the external pulser can be set to a higher frequency than the internal pulser without affecting the
calibration process, the external pulser circuit was added to the HFCB and the VSCM. Noise is measured using
external calibration charge pulses injected at low frequency in the absence of signal. The injected charge is shaped
and amplified in the analog circuitry to form an output signal. The voltage at the input of the discriminator is
measured over a certain number of triggers and the discriminator threshold determines whether or not the output
signal corresponds to a hit. The probability that the injected charge produces a hit depends on the setting of the
discriminator threshold. The average hit probability is measured by repeating the process of injecting charges and
counting the fraction of readout triggers that produced a hit. This measurement is repeated over a range of
threshold settings to produce an occupancy plot. 

The noise occupancy plots are produced by setting the pulser amplitude at fixed values and changing the comparator
thresholds. Each point of an occupancy plot for a fixed value of injected charge represents the percentage of time
that the comparator fires for a certain value of the threshold settings. In Fig.~\ref{fig:threshold-scan} three
occupancy plots taken at different pulser amplitudes for a single channel are presented. The noise occupancy as a
function of threshold in case of Gaussian noise is fit by the complementary error function and the probability $p$
of surpassing a threshold $\tau$ is given by Eq.(\ref{eq:erfc}). 

\begin{equation}
  p(\tau)=\frac{1}{2} erfc \left( \frac{\tau}{\sqrt{2}\sigma} \right)
  \label{eq:erfc},
\end{equation}
where $\sigma$ is the standard deviation of the Gaussian distribution. In the presence of the calibration signal,
the occupancy plot shifts to higher threshold values proportionally to the calibration amplitude.

In between the high and low threshold regions, the occupancy histogram for each SVT channel is well described by
an error function, or S-curve, producing a mean value (discriminator threshold) and standard deviation (noise). The
parameters for the mean and $\sigma$ of the $erfc$ fits are shown. The conversion to mV is performed considering
the width of one DAC bin (3.5~mV). The injected charge is calculated using the nominal value for the FSSR2 injection
capacitance of 40~fF. The threshold at 50\% occupancy is plotted against the input charge, resulting in a response
curve, of which the slope is called the gain $G$, measured in mV/fC. The response plot shows the linear dependence
of the output pulse height on the input charge in the operation region of the preamplifier. The middle point
corresponds to the charge deposited in the sensor by a minimum-ionizing particle (MIP). The slope and the offset
parameters of the linear fit are shown. The input noise is obtained by dividing the output noise by the gain using:

\begin{equation}
  ENC[e^{-}]=6242[e^{-}/{\rm fC}]\frac{\sigma[{\rm mV}]}{G[{\rm mV/fC}]}
  \label{eq:enc}
\end{equation}

\begin{figure}[hbt] 
	\centering 
	\includegraphics[width=1.0\columnwidth,keepaspectratio]{thrdisp.jpg}
	\caption{Typical threshold dispersion within an FSSR2 ASIC chip.}
	\label{fig:thrdisp}
\end{figure}

The threshold charge must be the same across the channels in the detector, otherwise the track-finding algorithms
\cite{recon-nim} would be biased by the potential extra hits. Any spread in the response among the different
channels of a chip results in a spread of the efficiency and noise occupancy that degrades the effective performance.
This leads to a requirement that the channel-to-channel variations in threshold and noise are kept to a minimum.
Threshold dispersion is defined to be the standard deviation of the distribution of means obtained from the
parameters of the complementary error function fit.

The noise for each individual detector channel is measured and the values are used by the zero-suppression
algorithms implemented in the core logic of the FSSR2 and by the calibration procedures to identify defective
channels. A comparison of the noise for the 33~cm strips demonstrates that the threshold spread is negligible
compared  to the noise and does not affect the efficiency and noise occupancy (see Fig.~\ref{fig:thrdisp}). The
threshold dispersion agrees with expectations for the FSSR2 chip for the chosen settings.

The SVT calibration parameters are stored in the CLAS12 calibration database (CCDB)~\cite{recon-nim}. The
channel calibration table has columns corresponding to sector, layer, chip identifier, mean, channel status (good,
noisy, open, dead, or masked), ENC, gain, offset, V$_{t50}$ (threshold at 50\% occupancy), and the threshold.
There are 21504 rows in the channel calibration table. The ENC and gain are calculated using a calibration amplitude
equal to 100 DAC counts (injected charge 4~fC). The chip calibration table has columns corresponding to layer, sector,
chip identifier, ENC (electrons), gain (mV/fC), offset (mV), the threshold at 50$\%$ occupancy (V$_{t50}$, mV),
threshold dispersion (electrons), chip gain (low, high), BLR mode (off, on), BCO time (ns), shaper time (ns), 8 ADC
thresholds in DAC. There are 168 rows in the chip calibration table. 

